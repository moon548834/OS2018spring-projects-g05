\documentclass{article}

\usepackage{amsmath,amsfonts,latexsym,graphicx}
\usepackage{fullpage,color}
\usepackage{setspace}
\usepackage{float}

\usepackage{tocloft}
\setlength{\cftbeforesecskip}{10pt}
\renewcommand{\cftbeforesubsecskip}{4pt}
\renewcommand{\cftbeforesubsubsecskip}{4pt}
\renewcommand\cftdotsep{2}
\setcounter{secnumdepth}{4}
\setcounter{tocdepth}{4}

\usepackage{fancyhdr}
\usepackage{hyperref}

\usepackage{xeCJK}
\setCJKmainfont{Songti SC}
\setCJKsansfont{Songti SC}
\setCJKmonofont{Songti SC}

\usepackage{indentfirst}
\setlength{\parindent}{2em}

\usepackage[linesnumbered,boxed,ruled,vlined]{algorithm2e}

\begin{document}
\begin{titlepage}
\fancyhead[CH]{}

\hspace{3.0cm}
\begin{center}
\vfill
% Upper part of the page

\textsc{\LARGE Tsinghua University}\\[1.5cm]

\textsc{\Large Computer Organization Experimentation}\\[0.5cm]


% Title
\rule[0.75\baselineskip]{0.75\textwidth}{1pt}

{ \huge \bfseries REPORT\_NAME\_HERE}\\[0.4cm]

\rule[20\baselineskip]{0.75\textwidth}{1pt}

% Author and supervisor
\begin{minipage}{0.4\textwidth}
\begin{flushleft} \large
\emph{Author:}\\
Shizhi \textsc{Tang}

Xihang \textsc{Liu}

Zixi \textsc{Cai}
\end{flushleft}
\end{minipage}
\begin{minipage}{0.4\textwidth}
\begin{flushright} \large
\emph{Supervisor:} \\
Yuxiang \textsc{Zhang}

Prof.~Weidong \textsc{Liu}
\end{flushright}
\end{minipage}

\vfill
\vspace{3.0cm}
% Bottom of the page
{\large \today}

\end{center}

\end{titlepage}
\setcounter{page}{2}
\tableofcontents
\newpage
\begin{spacing}{1.4}

\section{对象模板}
\subsection{公式}
$$f(x)=x^2$$
$$g(x)=x_3$$
\begin{eqnarray}
\begin{split}
h(x)&=x^3+3x^2+3x+1 \\
&=(x+1)^3
\end{split}
\end{eqnarray}

\subsection{角注}
\indent 利用线性网络求解非线性问题\footnote{This is a footnode}是一种特立独行的思路。

\noindent 这一行\footnote{This is also a footnode}没有缩进。
\subsection{表格}
\begin{table}[!htb]
\begin{center}
\begin{tabular*}{15cm}{c|c|c|c}  
\hline  
\textbf{格式}&\textbf{arg1}&\textbf{arg2}&\textbf{描述} \\
\hline
list & None & None &读取和反汇编从当前地址开始的10条指令 \\
\hline
list[count] & count & None & etc \\
\hline 
\end{tabular*}  
\caption{这是可爱的表格名}
\end{center}
\end{table}

\subsection{指令序列}
\begin{table}[!htb]
\begin{center}
\begin{tabular*}{15cm}{lll}  
\hline  
\$ r\\
Running...\\
Debugger reset.\\
\$ c\\
Operating system not started.\\
\hline  
\end{tabular*}  
\end{center}
\end{table}

\subsection{代码/伪代码}

\begin{algorithm}[H]
\caption{An Algorithm}
\label{algo2}
Initially, let $\mathcal{L} = 0$\;
\While{$a>b$}{
    \For{$m = 1,2,\ldots, s+1$}{
        $\Theta=\Xi$\;
    }
   $\clubsuit\diamondsuit\heartsuit\spadesuit$\;
    \If{$b<a$}{ 
        Do nothing\;
    }
}%while
{\bf Return} $\bigoplus$\;
\end{algorithm}

\subsection{bullet sequence}
\begin{itemize}
\item condition a
\item condition b
\item[bba] condition c
\end{itemize}

\renewcommand\refname{\large 参考文献}
\begin{thebibliography}{}  
\bibitem[显示符号]{引用标签} Book Title, Author  
\end{thebibliography}

\section{其他注意事项}
\end{spacing}
\end{document}